\documentclass{article}

\usepackage{hyperref}
\usepackage{titlesec}
\usepackage{titling}
\usepackage[margin = 1.25in]{geometry}

\titleformat{\section}
{\large}
{}
{0em}
{\bfseries}[\titlerule]

\titleformat{\subsection}
{\bfseries}
{$\bullet$}
{1em}
{}

\titleformat{\subsubsection} [runin]
{\bfseries}
{}
{0em}
{}

\titlespacing{\subsubsection}
{0em}{0.5em}{0.5em}

\renewcommand{\maketitle}{
\begin{center}
{\huge\bfseries
\theauthor}

\vspace{.25em}

Email: xiaoyu.sun@rutgers.edu\\
\end{center}
}

% --------------------------------------------------------
\begin{document}
\title{CV}
\author{Xiaoyu Sun}
\maketitle

\section{Education}
\textbf{Rutgers, The State University of New Jersey} \space 
\hfill{New Brunswick, NJ}\\
Computer Science, B.S. \space
\hfill{Sept 2018 - Present}\\
\textbf{Cumulative GPA: 3.82/4.00}\\
\textbf{Major GPA: 3.95/4.00}


\section{Teaching Experience}
\textbf{Rutgers Office for Diversity and Academic Success in the Sciences}
\hfill{New Brunswick, NJ}\\
Recitation Instructor
\hfill{Sept 2020 - Present}
\begin{itemize}
  \item Fall 2020 - General Physics II
  \begin{itemize}
  	\item Topics included: Electricity, Electric Circuits, Electromagnetism, Optics, Special Relativity, and Radioactivity.
     \item Recorded recitation videos each week explaining homework questions.
     \item Held three hours of review session each week for 30+ students.
 	 \item Held weekly office hours.
  \end{itemize}
\end{itemize}
\begin{itemize}
  \item Spring 2021 - General Physics I
  \begin{itemize}
  	\item Topics included: Newton's Law,  Work and Energy,  Fluids,  Heat,  Thermodynamics,  etc.
     \item Held  recitation each week explaining homework questions.
  \end{itemize}
\end{itemize}


\section{Research Experience}
\textbf{Rutgers Department of Civil \& Environmental Engineering}
\hfill{New Brunswick, NJ}\\
Research Assistant to Prof. Xiang Liu
\hfill{Sept 2020 - Present}
\begin{itemize}
  \itemsep0em
  \item Conducted research on a computer vision project related to railway safety.  The goal is to improve railway safety by adapting machine learning tools to do passenger recognition. 
  
\end{itemize}



\section{Professional Experience}
\textbf{Rutgers Undergraduate Research Journal}
\hfill{New Brunswick, NJ}\\
Co-founder, Editor
\hfill{Sept 2019 - Sept 2020}
\begin{itemize}
  \itemsep0em
  \item Founded the university-wide undergraduate research journal.
  \item Provided training sessions for new members to familiar with the review process.
  \item Reviewed papers in different fields as a reviewer, and given suggestions on logic flow and construction.
  \item Assisted professors in reviewing and giving reports to authors with feedback on the research content.
\end{itemize}


\section{Research Projects}
\textbf{Fast Trajectory Replanning}
\hfill{Summer 2020}\\
Rutgers University
\begin{itemize}
  \itemsep0em
  \item Implemented path planning with Python. Using several modified A* algorithms to find the shortest path for an agent to move toward a goal in an unknown environment with randomly generalized obstacles.
\end{itemize}

\noindent \textbf{Face and Digit Classification}
\hfill{Summer 2020}\\
Rutgers University
\begin{itemize}
  \itemsep0em
  \item Designed two classifiers, a Naive Bayes classifier, and a Perceptron classifier. Applied both classifiers to do each of the two tasks: digit recognition and face detection. Obtained over 60\% accuracy for digit recognition and 70\% for facial detection.
\end{itemize}

\noindent \textbf{Linear Image Classifier}
\hfill{Spring 2020}\\
Rutgers University
\begin{itemize}
  \itemsep0em
  \item Built a linear image classifier from scratch in PyTorch using CIFAR10 dataset. Implemented both the forward pass and backward pass of the linear classifier without using PyTorch's autograd capabilities.
\end{itemize}

\noindent \textbf{Neural Machine Translation}
\hfill{Spring 2020}\\
Rutgers University
\begin{itemize}
  \itemsep0em
  \item Implemented neural machine translation (NMT) models using recurrent neural networks (RNN), long short-term memory (LSTM) with attention, and transformers.
\end{itemize}

\noindent \textbf{Reinforcement Learning}
\hfill{Spring 2020}\\
Rutgers University
\begin{itemize}
  \itemsep0em
  \item Implemented Deep Q-Networks (DQN) to train an agent to play atari pong game from OpenAI Gym environment.
\end{itemize}

\noindent \textbf{Transfer Learning for Covid-19 Detection from Chest X-Ray}
\hfill{Spring 2020}\\
Rutgers University
\begin{itemize}
  \itemsep0em
  \item  Adapted transfer learning to analyze the Chest X-ray from suspected cases.  With a tuned vgg-16 model, we had an accuracy of 90\% on average for Covid-19 detection on the test cases.
\end{itemize}

\section{Honors}
\begin{itemize}
	\item Dean's List, Rutgers University, New Brunswick, NJ \hfill{2018 - 2020}
	\item Phi Beta Kappa National Honor Society
\end{itemize}

\section{Skills}

\subsubsection{Programming Language:}

Python, Java, C, mySQL,  Javascript

\subsubsection{Framework:}

PyTorch

\subsubsection{Markup:}

LaTex, Markdown

\subsubsection{Language:}
English, Chinese
\end{document}